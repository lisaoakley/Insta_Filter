
% Default to the notebook output style

    


% Inherit from the specified cell style.




    
\documentclass[11pt]{article}

    
    
    \usepackage[T1]{fontenc}
    % Nicer default font than Computer Modern for most use cases
    \usepackage{palatino}

    % Basic figure setup, for now with no caption control since it's done
    % automatically by Pandoc (which extracts ![](path) syntax from Markdown).
    \usepackage{graphicx}
    % We will generate all images so they have a width \maxwidth. This means
    % that they will get their normal width if they fit onto the page, but
    % are scaled down if they would overflow the margins.
    \makeatletter
    \def\maxwidth{\ifdim\Gin@nat@width>\linewidth\linewidth
    \else\Gin@nat@width\fi}
    \makeatother
    \let\Oldincludegraphics\includegraphics
    % Set max figure width to be 80% of text width, for now hardcoded.
    \renewcommand{\includegraphics}[1]{\Oldincludegraphics[width=.8\maxwidth]{#1}}
    % Ensure that by default, figures have no caption (until we provide a
    % proper Figure object with a Caption API and a way to capture that
    % in the conversion process - todo).
    \usepackage{caption}
    \DeclareCaptionLabelFormat{nolabel}{}
    \captionsetup{labelformat=nolabel}

    \usepackage{adjustbox} % Used to constrain images to a maximum size 
    \usepackage{xcolor} % Allow colors to be defined
    \usepackage{enumerate} % Needed for markdown enumerations to work
    \usepackage{geometry} % Used to adjust the document margins
    \usepackage{amsmath} % Equations
    \usepackage{amssymb} % Equations
    \usepackage{textcomp} % defines textquotesingle
    % Hack from http://tex.stackexchange.com/a/47451/13684:
    \AtBeginDocument{%
        \def\PYZsq{\textquotesingle}% Upright quotes in Pygmentized code
    }
    \usepackage{upquote} % Upright quotes for verbatim code
    \usepackage{eurosym} % defines \euro
    \usepackage[mathletters]{ucs} % Extended unicode (utf-8) support
    \usepackage[utf8x]{inputenc} % Allow utf-8 characters in the tex document
    \usepackage{fancyvrb} % verbatim replacement that allows latex
    \usepackage{grffile} % extends the file name processing of package graphics 
                         % to support a larger range 
    % The hyperref package gives us a pdf with properly built
    % internal navigation ('pdf bookmarks' for the table of contents,
    % internal cross-reference links, web links for URLs, etc.)
    \usepackage{hyperref}
    \usepackage{longtable} % longtable support required by pandoc >1.10
    \usepackage{booktabs}  % table support for pandoc > 1.12.2
    \usepackage[normalem]{ulem} % ulem is needed to support strikethroughs (\sout)
                                % normalem makes italics be italics, not underlines
    

    
    
    % Colors for the hyperref package
    \definecolor{urlcolor}{rgb}{0,.145,.698}
    \definecolor{linkcolor}{rgb}{.71,0.21,0.01}
    \definecolor{citecolor}{rgb}{.12,.54,.11}

    % ANSI colors
    \definecolor{ansi-black}{HTML}{3E424D}
    \definecolor{ansi-black-intense}{HTML}{282C36}
    \definecolor{ansi-red}{HTML}{E75C58}
    \definecolor{ansi-red-intense}{HTML}{B22B31}
    \definecolor{ansi-green}{HTML}{00A250}
    \definecolor{ansi-green-intense}{HTML}{007427}
    \definecolor{ansi-yellow}{HTML}{DDB62B}
    \definecolor{ansi-yellow-intense}{HTML}{B27D12}
    \definecolor{ansi-blue}{HTML}{208FFB}
    \definecolor{ansi-blue-intense}{HTML}{0065CA}
    \definecolor{ansi-magenta}{HTML}{D160C4}
    \definecolor{ansi-magenta-intense}{HTML}{A03196}
    \definecolor{ansi-cyan}{HTML}{60C6C8}
    \definecolor{ansi-cyan-intense}{HTML}{258F8F}
    \definecolor{ansi-white}{HTML}{C5C1B4}
    \definecolor{ansi-white-intense}{HTML}{A1A6B2}

    % commands and environments needed by pandoc snippets
    % extracted from the output of `pandoc -s`
    \providecommand{\tightlist}{%
      \setlength{\itemsep}{0pt}\setlength{\parskip}{0pt}}
    \DefineVerbatimEnvironment{Highlighting}{Verbatim}{commandchars=\\\{\}}
    % Add ',fontsize=\small' for more characters per line
    \newenvironment{Shaded}{}{}
    \newcommand{\KeywordTok}[1]{\textcolor[rgb]{0.00,0.44,0.13}{\textbf{{#1}}}}
    \newcommand{\DataTypeTok}[1]{\textcolor[rgb]{0.56,0.13,0.00}{{#1}}}
    \newcommand{\DecValTok}[1]{\textcolor[rgb]{0.25,0.63,0.44}{{#1}}}
    \newcommand{\BaseNTok}[1]{\textcolor[rgb]{0.25,0.63,0.44}{{#1}}}
    \newcommand{\FloatTok}[1]{\textcolor[rgb]{0.25,0.63,0.44}{{#1}}}
    \newcommand{\CharTok}[1]{\textcolor[rgb]{0.25,0.44,0.63}{{#1}}}
    \newcommand{\StringTok}[1]{\textcolor[rgb]{0.25,0.44,0.63}{{#1}}}
    \newcommand{\CommentTok}[1]{\textcolor[rgb]{0.38,0.63,0.69}{\textit{{#1}}}}
    \newcommand{\OtherTok}[1]{\textcolor[rgb]{0.00,0.44,0.13}{{#1}}}
    \newcommand{\AlertTok}[1]{\textcolor[rgb]{1.00,0.00,0.00}{\textbf{{#1}}}}
    \newcommand{\FunctionTok}[1]{\textcolor[rgb]{0.02,0.16,0.49}{{#1}}}
    \newcommand{\RegionMarkerTok}[1]{{#1}}
    \newcommand{\ErrorTok}[1]{\textcolor[rgb]{1.00,0.00,0.00}{\textbf{{#1}}}}
    \newcommand{\NormalTok}[1]{{#1}}
    
    % Additional commands for more recent versions of Pandoc
    \newcommand{\ConstantTok}[1]{\textcolor[rgb]{0.53,0.00,0.00}{{#1}}}
    \newcommand{\SpecialCharTok}[1]{\textcolor[rgb]{0.25,0.44,0.63}{{#1}}}
    \newcommand{\VerbatimStringTok}[1]{\textcolor[rgb]{0.25,0.44,0.63}{{#1}}}
    \newcommand{\SpecialStringTok}[1]{\textcolor[rgb]{0.73,0.40,0.53}{{#1}}}
    \newcommand{\ImportTok}[1]{{#1}}
    \newcommand{\DocumentationTok}[1]{\textcolor[rgb]{0.73,0.13,0.13}{\textit{{#1}}}}
    \newcommand{\AnnotationTok}[1]{\textcolor[rgb]{0.38,0.63,0.69}{\textbf{\textit{{#1}}}}}
    \newcommand{\CommentVarTok}[1]{\textcolor[rgb]{0.38,0.63,0.69}{\textbf{\textit{{#1}}}}}
    \newcommand{\VariableTok}[1]{\textcolor[rgb]{0.10,0.09,0.49}{{#1}}}
    \newcommand{\ControlFlowTok}[1]{\textcolor[rgb]{0.00,0.44,0.13}{\textbf{{#1}}}}
    \newcommand{\OperatorTok}[1]{\textcolor[rgb]{0.40,0.40,0.40}{{#1}}}
    \newcommand{\BuiltInTok}[1]{{#1}}
    \newcommand{\ExtensionTok}[1]{{#1}}
    \newcommand{\PreprocessorTok}[1]{\textcolor[rgb]{0.74,0.48,0.00}{{#1}}}
    \newcommand{\AttributeTok}[1]{\textcolor[rgb]{0.49,0.56,0.16}{{#1}}}
    \newcommand{\InformationTok}[1]{\textcolor[rgb]{0.38,0.63,0.69}{\textbf{\textit{{#1}}}}}
    \newcommand{\WarningTok}[1]{\textcolor[rgb]{0.38,0.63,0.69}{\textbf{\textit{{#1}}}}}
    
    
    % Define a nice break command that doesn't care if a line doesn't already
    % exist.
    \def\br{\hspace*{\fill} \\* }
    % Math Jax compatability definitions
    \def\gt{>}
    \def\lt{<}
    % Document parameters
    \title{Instagram\_Reverse\_Engineer}
    
    
    

    % Pygments definitions
    
\makeatletter
\def\PY@reset{\let\PY@it=\relax \let\PY@bf=\relax%
    \let\PY@ul=\relax \let\PY@tc=\relax%
    \let\PY@bc=\relax \let\PY@ff=\relax}
\def\PY@tok#1{\csname PY@tok@#1\endcsname}
\def\PY@toks#1+{\ifx\relax#1\empty\else%
    \PY@tok{#1}\expandafter\PY@toks\fi}
\def\PY@do#1{\PY@bc{\PY@tc{\PY@ul{%
    \PY@it{\PY@bf{\PY@ff{#1}}}}}}}
\def\PY#1#2{\PY@reset\PY@toks#1+\relax+\PY@do{#2}}

\expandafter\def\csname PY@tok@cs\endcsname{\let\PY@it=\textit\def\PY@tc##1{\textcolor[rgb]{0.25,0.50,0.50}{##1}}}
\expandafter\def\csname PY@tok@w\endcsname{\def\PY@tc##1{\textcolor[rgb]{0.73,0.73,0.73}{##1}}}
\expandafter\def\csname PY@tok@gi\endcsname{\def\PY@tc##1{\textcolor[rgb]{0.00,0.63,0.00}{##1}}}
\expandafter\def\csname PY@tok@sb\endcsname{\def\PY@tc##1{\textcolor[rgb]{0.73,0.13,0.13}{##1}}}
\expandafter\def\csname PY@tok@kr\endcsname{\let\PY@bf=\textbf\def\PY@tc##1{\textcolor[rgb]{0.00,0.50,0.00}{##1}}}
\expandafter\def\csname PY@tok@bp\endcsname{\def\PY@tc##1{\textcolor[rgb]{0.00,0.50,0.00}{##1}}}
\expandafter\def\csname PY@tok@vi\endcsname{\def\PY@tc##1{\textcolor[rgb]{0.10,0.09,0.49}{##1}}}
\expandafter\def\csname PY@tok@nf\endcsname{\def\PY@tc##1{\textcolor[rgb]{0.00,0.00,1.00}{##1}}}
\expandafter\def\csname PY@tok@nl\endcsname{\def\PY@tc##1{\textcolor[rgb]{0.63,0.63,0.00}{##1}}}
\expandafter\def\csname PY@tok@no\endcsname{\def\PY@tc##1{\textcolor[rgb]{0.53,0.00,0.00}{##1}}}
\expandafter\def\csname PY@tok@nv\endcsname{\def\PY@tc##1{\textcolor[rgb]{0.10,0.09,0.49}{##1}}}
\expandafter\def\csname PY@tok@gt\endcsname{\def\PY@tc##1{\textcolor[rgb]{0.00,0.27,0.87}{##1}}}
\expandafter\def\csname PY@tok@kc\endcsname{\let\PY@bf=\textbf\def\PY@tc##1{\textcolor[rgb]{0.00,0.50,0.00}{##1}}}
\expandafter\def\csname PY@tok@nc\endcsname{\let\PY@bf=\textbf\def\PY@tc##1{\textcolor[rgb]{0.00,0.00,1.00}{##1}}}
\expandafter\def\csname PY@tok@gp\endcsname{\let\PY@bf=\textbf\def\PY@tc##1{\textcolor[rgb]{0.00,0.00,0.50}{##1}}}
\expandafter\def\csname PY@tok@s2\endcsname{\def\PY@tc##1{\textcolor[rgb]{0.73,0.13,0.13}{##1}}}
\expandafter\def\csname PY@tok@c1\endcsname{\let\PY@it=\textit\def\PY@tc##1{\textcolor[rgb]{0.25,0.50,0.50}{##1}}}
\expandafter\def\csname PY@tok@sc\endcsname{\def\PY@tc##1{\textcolor[rgb]{0.73,0.13,0.13}{##1}}}
\expandafter\def\csname PY@tok@mo\endcsname{\def\PY@tc##1{\textcolor[rgb]{0.40,0.40,0.40}{##1}}}
\expandafter\def\csname PY@tok@ow\endcsname{\let\PY@bf=\textbf\def\PY@tc##1{\textcolor[rgb]{0.67,0.13,1.00}{##1}}}
\expandafter\def\csname PY@tok@err\endcsname{\def\PY@bc##1{\setlength{\fboxsep}{0pt}\fcolorbox[rgb]{1.00,0.00,0.00}{1,1,1}{\strut ##1}}}
\expandafter\def\csname PY@tok@kd\endcsname{\let\PY@bf=\textbf\def\PY@tc##1{\textcolor[rgb]{0.00,0.50,0.00}{##1}}}
\expandafter\def\csname PY@tok@gr\endcsname{\def\PY@tc##1{\textcolor[rgb]{1.00,0.00,0.00}{##1}}}
\expandafter\def\csname PY@tok@gs\endcsname{\let\PY@bf=\textbf}
\expandafter\def\csname PY@tok@s\endcsname{\def\PY@tc##1{\textcolor[rgb]{0.73,0.13,0.13}{##1}}}
\expandafter\def\csname PY@tok@go\endcsname{\def\PY@tc##1{\textcolor[rgb]{0.53,0.53,0.53}{##1}}}
\expandafter\def\csname PY@tok@k\endcsname{\let\PY@bf=\textbf\def\PY@tc##1{\textcolor[rgb]{0.00,0.50,0.00}{##1}}}
\expandafter\def\csname PY@tok@m\endcsname{\def\PY@tc##1{\textcolor[rgb]{0.40,0.40,0.40}{##1}}}
\expandafter\def\csname PY@tok@se\endcsname{\let\PY@bf=\textbf\def\PY@tc##1{\textcolor[rgb]{0.73,0.40,0.13}{##1}}}
\expandafter\def\csname PY@tok@ne\endcsname{\let\PY@bf=\textbf\def\PY@tc##1{\textcolor[rgb]{0.82,0.25,0.23}{##1}}}
\expandafter\def\csname PY@tok@vg\endcsname{\def\PY@tc##1{\textcolor[rgb]{0.10,0.09,0.49}{##1}}}
\expandafter\def\csname PY@tok@sd\endcsname{\let\PY@it=\textit\def\PY@tc##1{\textcolor[rgb]{0.73,0.13,0.13}{##1}}}
\expandafter\def\csname PY@tok@ni\endcsname{\let\PY@bf=\textbf\def\PY@tc##1{\textcolor[rgb]{0.60,0.60,0.60}{##1}}}
\expandafter\def\csname PY@tok@s1\endcsname{\def\PY@tc##1{\textcolor[rgb]{0.73,0.13,0.13}{##1}}}
\expandafter\def\csname PY@tok@c\endcsname{\let\PY@it=\textit\def\PY@tc##1{\textcolor[rgb]{0.25,0.50,0.50}{##1}}}
\expandafter\def\csname PY@tok@kp\endcsname{\def\PY@tc##1{\textcolor[rgb]{0.00,0.50,0.00}{##1}}}
\expandafter\def\csname PY@tok@mf\endcsname{\def\PY@tc##1{\textcolor[rgb]{0.40,0.40,0.40}{##1}}}
\expandafter\def\csname PY@tok@cp\endcsname{\def\PY@tc##1{\textcolor[rgb]{0.74,0.48,0.00}{##1}}}
\expandafter\def\csname PY@tok@kt\endcsname{\def\PY@tc##1{\textcolor[rgb]{0.69,0.00,0.25}{##1}}}
\expandafter\def\csname PY@tok@sh\endcsname{\def\PY@tc##1{\textcolor[rgb]{0.73,0.13,0.13}{##1}}}
\expandafter\def\csname PY@tok@na\endcsname{\def\PY@tc##1{\textcolor[rgb]{0.49,0.56,0.16}{##1}}}
\expandafter\def\csname PY@tok@nt\endcsname{\let\PY@bf=\textbf\def\PY@tc##1{\textcolor[rgb]{0.00,0.50,0.00}{##1}}}
\expandafter\def\csname PY@tok@cpf\endcsname{\let\PY@it=\textit\def\PY@tc##1{\textcolor[rgb]{0.25,0.50,0.50}{##1}}}
\expandafter\def\csname PY@tok@gh\endcsname{\let\PY@bf=\textbf\def\PY@tc##1{\textcolor[rgb]{0.00,0.00,0.50}{##1}}}
\expandafter\def\csname PY@tok@cm\endcsname{\let\PY@it=\textit\def\PY@tc##1{\textcolor[rgb]{0.25,0.50,0.50}{##1}}}
\expandafter\def\csname PY@tok@gu\endcsname{\let\PY@bf=\textbf\def\PY@tc##1{\textcolor[rgb]{0.50,0.00,0.50}{##1}}}
\expandafter\def\csname PY@tok@ch\endcsname{\let\PY@it=\textit\def\PY@tc##1{\textcolor[rgb]{0.25,0.50,0.50}{##1}}}
\expandafter\def\csname PY@tok@gd\endcsname{\def\PY@tc##1{\textcolor[rgb]{0.63,0.00,0.00}{##1}}}
\expandafter\def\csname PY@tok@ss\endcsname{\def\PY@tc##1{\textcolor[rgb]{0.10,0.09,0.49}{##1}}}
\expandafter\def\csname PY@tok@mb\endcsname{\def\PY@tc##1{\textcolor[rgb]{0.40,0.40,0.40}{##1}}}
\expandafter\def\csname PY@tok@nn\endcsname{\let\PY@bf=\textbf\def\PY@tc##1{\textcolor[rgb]{0.00,0.00,1.00}{##1}}}
\expandafter\def\csname PY@tok@ge\endcsname{\let\PY@it=\textit}
\expandafter\def\csname PY@tok@si\endcsname{\let\PY@bf=\textbf\def\PY@tc##1{\textcolor[rgb]{0.73,0.40,0.53}{##1}}}
\expandafter\def\csname PY@tok@kn\endcsname{\let\PY@bf=\textbf\def\PY@tc##1{\textcolor[rgb]{0.00,0.50,0.00}{##1}}}
\expandafter\def\csname PY@tok@nd\endcsname{\def\PY@tc##1{\textcolor[rgb]{0.67,0.13,1.00}{##1}}}
\expandafter\def\csname PY@tok@sx\endcsname{\def\PY@tc##1{\textcolor[rgb]{0.00,0.50,0.00}{##1}}}
\expandafter\def\csname PY@tok@vc\endcsname{\def\PY@tc##1{\textcolor[rgb]{0.10,0.09,0.49}{##1}}}
\expandafter\def\csname PY@tok@mi\endcsname{\def\PY@tc##1{\textcolor[rgb]{0.40,0.40,0.40}{##1}}}
\expandafter\def\csname PY@tok@nb\endcsname{\def\PY@tc##1{\textcolor[rgb]{0.00,0.50,0.00}{##1}}}
\expandafter\def\csname PY@tok@mh\endcsname{\def\PY@tc##1{\textcolor[rgb]{0.40,0.40,0.40}{##1}}}
\expandafter\def\csname PY@tok@o\endcsname{\def\PY@tc##1{\textcolor[rgb]{0.40,0.40,0.40}{##1}}}
\expandafter\def\csname PY@tok@sr\endcsname{\def\PY@tc##1{\textcolor[rgb]{0.73,0.40,0.53}{##1}}}
\expandafter\def\csname PY@tok@il\endcsname{\def\PY@tc##1{\textcolor[rgb]{0.40,0.40,0.40}{##1}}}

\def\PYZbs{\char`\\}
\def\PYZus{\char`\_}
\def\PYZob{\char`\{}
\def\PYZcb{\char`\}}
\def\PYZca{\char`\^}
\def\PYZam{\char`\&}
\def\PYZlt{\char`\<}
\def\PYZgt{\char`\>}
\def\PYZsh{\char`\#}
\def\PYZpc{\char`\%}
\def\PYZdl{\char`\$}
\def\PYZhy{\char`\-}
\def\PYZsq{\char`\'}
\def\PYZdq{\char`\"}
\def\PYZti{\char`\~}
% for compatibility with earlier versions
\def\PYZat{@}
\def\PYZlb{[}
\def\PYZrb{]}
\makeatother


    % Exact colors from NB
    \definecolor{incolor}{rgb}{0.0, 0.0, 0.5}
    \definecolor{outcolor}{rgb}{0.545, 0.0, 0.0}



    
    % Prevent overflowing lines due to hard-to-break entities
    \sloppy 
    % Setup hyperref package
    \hypersetup{
      breaklinks=true,  % so long urls are correctly broken across lines
      colorlinks=true,
      urlcolor=urlcolor,
      linkcolor=linkcolor,
      citecolor=citecolor,
      }
    % Slightly bigger margins than the latex defaults
    
    \geometry{verbose,tmargin=1in,bmargin=1in,lmargin=1in,rmargin=1in}
    
    

    \begin{document}
    
    
    \maketitle
    
    

    
    \section{Assignment 3 - Instagram Reverse
Engineer}\label{assignment-3---instagram-reverse-engineer}

    \subsection{Part A - Abstract}\label{part-a---abstract}

    This research intends to reverse engineer Instagram's Claredon filter
and apply it to some of our own images. We start by considering what
might be the most reasonable areas to look in. We then investigate the
histograms of original and filtered images, the pairwise differences
between the two, and some implementations of sharpening, color
alteration, and contrast to find a procedure that would achieve a close
match to the Claredon filter's effects. Although our end result is not
very visually similar to the Claredon filter, we can conclude that some
sharpening and contrast enhancement is used in Instagram's Claredon
filter.

    \subsection{Part B and C - Methods and
Code}\label{part-b-and-c---methods-and-code}

    \begin{Verbatim}[commandchars=\\\{\}]
{\color{incolor}In [{\color{incolor}336}]:} \PY{o}{\PYZpc{}}\PY{k}{matplotlib} inline
          \PY{k+kn}{from} \PY{n+nn}{matplotlib}\PY{n+nn}{.}\PY{n+nn}{axes} \PY{k}{import} \PY{n}{Subplot}
          \PY{k+kn}{import} \PY{n+nn}{matplotlib}\PY{n+nn}{.}\PY{n+nn}{pyplot} \PY{k}{as} \PY{n+nn}{plt}
          \PY{k+kn}{import} \PY{n+nn}{matplotlib}\PY{n+nn}{.}\PY{n+nn}{image} \PY{k}{as} \PY{n+nn}{mpimg}
          \PY{k+kn}{import} \PY{n+nn}{numpy} \PY{k}{as} \PY{n+nn}{np}
          \PY{k+kn}{from} \PY{n+nn}{numpy} \PY{k}{import} \PY{n}{ndarray}
          \PY{k+kn}{import} \PY{n+nn}{skimage}\PY{n+nn}{.}\PY{n+nn}{io} 
          \PY{k+kn}{from} \PY{n+nn}{skimage} \PY{k}{import} \PY{n}{img\PYZus{}as\PYZus{}float}\PY{p}{,} \PY{n}{img\PYZus{}as\PYZus{}ubyte}
          \PY{k+kn}{from} \PY{n+nn}{skimage} \PY{k}{import} \PY{n}{color}
          \PY{k+kn}{from} \PY{n+nn}{scipy}\PY{n+nn}{.}\PY{n+nn}{ndimage} \PY{k}{import} \PY{n}{convolve}
          \PY{k+kn}{import} \PY{n+nn}{skimage}\PY{n+nn}{.}\PY{n+nn}{filters} \PY{k}{as} \PY{n+nn}{filters}
          \PY{k+kn}{from} \PY{n+nn}{PIL} \PY{k}{import} \PY{n}{Image}
          \PY{k+kn}{from} \PY{n+nn}{PIL} \PY{k}{import} \PY{n}{ImageEnhance}
          \PY{k+kn}{from} \PY{n+nn}{PIL} \PY{k}{import} \PY{n}{ImageFilter}
          \PY{k+kn}{import} \PY{n+nn}{glob}
\end{Verbatim}

    \begin{Verbatim}[commandchars=\\\{\}]
{\color{incolor}In [{\color{incolor}138}]:} \PY{c+c1}{\PYZsh{} Set defaults}
          \PY{n}{plt}\PY{o}{.}\PY{n}{rcParams}\PY{p}{[}\PY{l+s+s1}{\PYZsq{}}\PY{l+s+s1}{image.cmap}\PY{l+s+s1}{\PYZsq{}}\PY{p}{]} \PY{o}{=} \PY{l+s+s1}{\PYZsq{}}\PY{l+s+s1}{gray}\PY{l+s+s1}{\PYZsq{}} \PY{c+c1}{\PYZsh{} Display grayscale images in... grayscale.}
          \PY{n}{plt}\PY{o}{.}\PY{n}{rcParams}\PY{p}{[}\PY{l+s+s1}{\PYZsq{}}\PY{l+s+s1}{image.interpolation}\PY{l+s+s1}{\PYZsq{}}\PY{p}{]} \PY{o}{=} \PY{l+s+s1}{\PYZsq{}}\PY{l+s+s1}{none}\PY{l+s+s1}{\PYZsq{}} \PY{c+c1}{\PYZsh{} Use nearest\PYZhy{}neighbour}
          \PY{n}{plt}\PY{o}{.}\PY{n}{rcParams}\PY{p}{[}\PY{l+s+s1}{\PYZsq{}}\PY{l+s+s1}{figure.figsize}\PY{l+s+s1}{\PYZsq{}}\PY{p}{]} \PY{o}{=} \PY{l+m+mi}{10}\PY{p}{,} \PY{l+m+mi}{10}
\end{Verbatim}

    \begin{Verbatim}[commandchars=\\\{\}]
{\color{incolor}In [{\color{incolor}197}]:} \PY{c+c1}{\PYZsh{} Import test images as ubytes}
          \PY{n}{imgpaths} \PY{o}{=} \PY{n}{glob}\PY{o}{.}\PY{n}{glob}\PY{p}{(}\PY{l+s+s2}{\PYZdq{}}\PY{l+s+s2}{./images/*.JPG}\PY{l+s+s2}{\PYZdq{}}\PY{p}{)}
          \PY{n}{imgset} \PY{o}{=} \PY{p}{[}\PY{n}{img\PYZus{}as\PYZus{}ubyte}\PY{p}{(}\PY{n}{mpimg}\PY{o}{.}\PY{n}{imread}\PY{p}{(}\PY{n}{x}\PY{p}{)}\PY{p}{)} \PY{k}{for} \PY{n}{x} \PY{o+ow}{in} \PY{n}{imgpaths}\PY{p}{]}
          \PY{c+c1}{\PYZsh{} Import test images as floats}
          \PY{n}{imgpaths2} \PY{o}{=} \PY{n}{glob}\PY{o}{.}\PY{n}{glob}\PY{p}{(}\PY{l+s+s2}{\PYZdq{}}\PY{l+s+s2}{./images/*.JPG}\PY{l+s+s2}{\PYZdq{}}\PY{p}{)}
          \PY{n}{imgset2} \PY{o}{=} \PY{p}{[}\PY{n}{img\PYZus{}as\PYZus{}float}\PY{p}{(}\PY{n}{mpimg}\PY{o}{.}\PY{n}{imread}\PY{p}{(}\PY{n}{x}\PY{p}{)}\PY{p}{)} \PY{k}{for} \PY{n}{x} \PY{o+ow}{in} \PY{n}{imgpaths2}\PY{p}{]}
\end{Verbatim}

    \subsubsection{i. Analyze the
histograms}\label{i.-analyze-the-histograms}

    \begin{Verbatim}[commandchars=\\\{\}]
{\color{incolor}In [{\color{incolor}23}]:} \PY{c+c1}{\PYZsh{} Plots a histogram of the image, splitting into individual channels if necessary.}
         \PY{k}{def} \PY{n+nf}{plot\PYZus{}multichannel\PYZus{}histo}\PY{p}{(}\PY{n}{img}\PY{p}{)}\PY{p}{:}
             \PY{k}{if} \PY{n}{img}\PY{o}{.}\PY{n}{ndim} \PY{o}{==} \PY{l+m+mi}{2}\PY{p}{:} \PY{c+c1}{\PYZsh{} plot grayscale histo}
                 \PY{n}{plt}\PY{o}{.}\PY{n}{hist}\PY{p}{(}\PY{n}{img}\PY{o}{.}\PY{n}{flatten}\PY{p}{(}\PY{p}{)}\PY{p}{,} \PY{l+m+mi}{256}\PY{p}{,}  \PY{n+nb}{range}\PY{o}{=}\PY{p}{(}\PY{l+m+mi}{0}\PY{p}{,}\PY{l+m+mi}{255}\PY{p}{)}\PY{p}{,} \PY{n}{color}\PY{o}{=}\PY{l+s+s1}{\PYZsq{}}\PY{l+s+s1}{k}\PY{l+s+s1}{\PYZsq{}}\PY{p}{,} \PY{n}{histtype}\PY{o}{=}\PY{l+s+s1}{\PYZsq{}}\PY{l+s+s1}{step}\PY{l+s+s1}{\PYZsq{}}\PY{p}{)}
             \PY{k}{elif} \PY{n}{img}\PY{o}{.}\PY{n}{ndim} \PY{o}{==} \PY{l+m+mi}{3}\PY{p}{:} \PY{c+c1}{\PYZsh{} print rgb histo}
                 \PY{n}{plt}\PY{o}{.}\PY{n}{hist}\PY{p}{(}\PY{n}{img}\PY{o}{.}\PY{n}{reshape}\PY{p}{(}\PY{o}{\PYZhy{}}\PY{l+m+mi}{1}\PY{p}{,}\PY{l+m+mi}{3}\PY{p}{)}\PY{p}{,} \PY{l+m+mi}{256}\PY{p}{,}  \PY{n+nb}{range}\PY{o}{=}\PY{p}{(}\PY{l+m+mi}{0}\PY{p}{,}\PY{l+m+mi}{255}\PY{p}{)}\PY{p}{,} \PY{n}{color}\PY{o}{=}\PY{p}{[}\PY{l+s+s1}{\PYZsq{}}\PY{l+s+s1}{r}\PY{l+s+s1}{\PYZsq{}}\PY{p}{,}\PY{l+s+s1}{\PYZsq{}}\PY{l+s+s1}{g}\PY{l+s+s1}{\PYZsq{}}\PY{p}{,}\PY{l+s+s1}{\PYZsq{}}\PY{l+s+s1}{b}\PY{l+s+s1}{\PYZsq{}}\PY{p}{]}\PY{p}{,}\PY{n}{histtype}\PY{o}{=}\PY{l+s+s1}{\PYZsq{}}\PY{l+s+s1}{step}\PY{l+s+s1}{\PYZsq{}}\PY{p}{)}
             \PY{k}{else}\PY{p}{:} \PY{c+c1}{\PYZsh{} Not an image}
                 \PY{n+nb}{print}\PY{p}{(}\PY{l+s+s2}{\PYZdq{}}\PY{l+s+s2}{Must pass a valid RGB or grayscale image}\PY{l+s+s2}{\PYZdq{}}\PY{p}{)}
             \PY{n}{plt}\PY{o}{.}\PY{n}{xlim}\PY{p}{(}\PY{p}{[}\PY{l+m+mi}{0}\PY{p}{,}\PY{l+m+mi}{255}\PY{p}{]}\PY{p}{)}
\end{Verbatim}

    \begin{Verbatim}[commandchars=\\\{\}]
{\color{incolor}In [{\color{incolor}355}]:} \PY{c+c1}{\PYZsh{} Apply histogram to original images and images filtered with Instagram\PYZsq{}s Claredon filter}
          \PY{k}{for} \PY{n}{i}\PY{p}{,}\PY{n}{img} \PY{o+ow}{in} \PY{n+nb}{enumerate}\PY{p}{(}\PY{n}{imgset}\PY{p}{)}\PY{p}{:}
              \PY{k}{if} \PY{p}{(}\PY{l+m+mi}{1} \PY{o}{==} \PY{n}{i}\PY{o}{\PYZpc{}}\PY{k}{2}):
                  \PY{n}{imtitle} \PY{o}{=} \PY{l+s+s1}{\PYZsq{}}\PY{l+s+s1}{Original}\PY{l+s+s1}{\PYZsq{}}
              \PY{k}{else}\PY{p}{:}
                  \PY{n}{imtitle} \PY{o}{=} \PY{l+s+s1}{\PYZsq{}}\PY{l+s+s1}{Claredon}\PY{l+s+s1}{\PYZsq{}}
              \PY{n}{plt}\PY{o}{.}\PY{n}{figure}\PY{p}{(}\PY{p}{)}
              \PY{n}{plt}\PY{o}{.}\PY{n}{subplot}\PY{p}{(}\PY{l+m+mi}{1}\PY{p}{,} \PY{l+m+mi}{2}\PY{p}{,} \PY{l+m+mi}{1}\PY{p}{)}
              \PY{n}{plt}\PY{o}{.}\PY{n}{title}\PY{p}{(}\PY{n}{imtitle}\PY{p}{)}
              \PY{n}{plt}\PY{o}{.}\PY{n}{imshow}\PY{p}{(}\PY{n}{img}\PY{p}{,} \PY{n}{cmap}\PY{o}{=}\PY{l+s+s1}{\PYZsq{}}\PY{l+s+s1}{gray}\PY{l+s+s1}{\PYZsq{}}\PY{p}{)}
              \PY{n}{plt}\PY{o}{.}\PY{n}{subplot}\PY{p}{(}\PY{l+m+mi}{1}\PY{p}{,} \PY{l+m+mi}{2}\PY{p}{,} \PY{l+m+mi}{2}\PY{p}{)}
              \PY{n}{plt}\PY{o}{.}\PY{n}{title}\PY{p}{(}\PY{n}{imtitle} \PY{o}{+} \PY{l+s+s1}{\PYZsq{}}\PY{l+s+s1}{ Histo}\PY{l+s+s1}{\PYZsq{}}\PY{p}{)}
              \PY{n}{plot\PYZus{}multichannel\PYZus{}histo}\PY{p}{(}\PY{n}{img}\PY{p}{)}
              
\end{Verbatim}

    \begin{center}
    \adjustimage{max size={0.9\linewidth}{0.9\paperheight}}{Instagram_Reverse_Engineer_files/Instagram_Reverse_Engineer_9_0.png}
    \end{center}
    { \hspace*{\fill} \\}
    
    \begin{center}
    \adjustimage{max size={0.9\linewidth}{0.9\paperheight}}{Instagram_Reverse_Engineer_files/Instagram_Reverse_Engineer_9_1.png}
    \end{center}
    { \hspace*{\fill} \\}
    
    \begin{center}
    \adjustimage{max size={0.9\linewidth}{0.9\paperheight}}{Instagram_Reverse_Engineer_files/Instagram_Reverse_Engineer_9_2.png}
    \end{center}
    { \hspace*{\fill} \\}
    
    \begin{center}
    \adjustimage{max size={0.9\linewidth}{0.9\paperheight}}{Instagram_Reverse_Engineer_files/Instagram_Reverse_Engineer_9_3.png}
    \end{center}
    { \hspace*{\fill} \\}
    
    \begin{center}
    \adjustimage{max size={0.9\linewidth}{0.9\paperheight}}{Instagram_Reverse_Engineer_files/Instagram_Reverse_Engineer_9_4.png}
    \end{center}
    { \hspace*{\fill} \\}
    
    \begin{center}
    \adjustimage{max size={0.9\linewidth}{0.9\paperheight}}{Instagram_Reverse_Engineer_files/Instagram_Reverse_Engineer_9_5.png}
    \end{center}
    { \hspace*{\fill} \\}
    
    Here we look at the histograms of 3 images and their filtered
counterpart. There appears to be general raise in contrast, with an
emphasis on blue values in particular.

    \subsection{ii. Analyze pairwise
differences}\label{ii.-analyze-pairwise-differences}

    \begin{Verbatim}[commandchars=\\\{\}]
{\color{incolor}In [{\color{incolor}46}]:} \PY{c+c1}{\PYZsh{} Downsample an image by skipping indicies}
         \PY{k}{def} \PY{n+nf}{decimate\PYZus{}image}\PY{p}{(}\PY{n}{img}\PY{p}{,} \PY{n}{skip}\PY{p}{)}\PY{p}{:}
              \PY{k}{return} \PY{n}{img}\PY{p}{[}\PY{p}{:}\PY{p}{:}\PY{n}{skip}\PY{p}{,}\PY{p}{:}\PY{p}{:}\PY{n}{skip}\PY{p}{]}
\end{Verbatim}

    \begin{Verbatim}[commandchars=\\\{\}]
{\color{incolor}In [{\color{incolor}47}]:} \PY{c+c1}{\PYZsh{} Find the absolute difference between two images. }
         \PY{c+c1}{\PYZsh{} Crops to the shared region between images.}
         \PY{k}{def} \PY{n+nf}{find\PYZus{}pairwise\PYZus{}difference}\PY{p}{(}\PY{n}{img\PYZus{}a}\PY{p}{,} \PY{n}{img\PYZus{}b}\PY{p}{)}\PY{p}{:}
             \PY{n}{subset} \PY{o}{=} \PY{n}{np}\PY{o}{.}\PY{n}{minimum}\PY{p}{(}\PY{n}{img\PYZus{}a}\PY{o}{.}\PY{n}{shape}\PY{p}{,} \PY{n}{img\PYZus{}b}\PY{o}{.}\PY{n}{shape}\PY{p}{)}
             \PY{n}{img\PYZus{}a\PYZus{}subset} \PY{o}{=} \PY{n}{img\PYZus{}a}\PY{p}{[}\PY{p}{:}\PY{n}{subset}\PY{p}{[}\PY{l+m+mi}{0}\PY{p}{]}\PY{p}{,} \PY{p}{:}\PY{n}{subset}\PY{p}{[}\PY{l+m+mi}{1}\PY{p}{]}\PY{p}{]}
             \PY{n}{img\PYZus{}b\PYZus{}subset} \PY{o}{=} \PY{n}{img\PYZus{}b}\PY{p}{[}\PY{p}{:}\PY{n}{subset}\PY{p}{[}\PY{l+m+mi}{0}\PY{p}{]}\PY{p}{,} \PY{p}{:}\PY{n}{subset}\PY{p}{[}\PY{l+m+mi}{1}\PY{p}{]}\PY{p}{]}
             \PY{k}{return} \PY{n}{img\PYZus{}a\PYZus{}subset}\PY{p}{,} \PY{n}{img\PYZus{}b\PYZus{}subset}\PY{p}{,} \PY{n}{np}\PY{o}{.}\PY{n}{abs}\PY{p}{(}\PY{n}{img\PYZus{}a\PYZus{}subset} \PY{o}{\PYZhy{}} \PY{n}{img\PYZus{}b\PYZus{}subset}\PY{p}{)}
\end{Verbatim}

    \begin{Verbatim}[commandchars=\\\{\}]
{\color{incolor}In [{\color{incolor}109}]:} \PY{c+c1}{\PYZsh{} Compare each image, pairwise, with its claredon counterpart.}
          \PY{k}{for} \PY{n}{i} \PY{o+ow}{in} \PY{n+nb}{range}\PY{p}{(}\PY{n+nb}{len}\PY{p}{(}\PY{n}{imgset}\PY{p}{)}\PY{p}{)}\PY{p}{:}
              \PY{k}{if} \PY{p}{(}\PY{l+m+mi}{0}\PY{o}{==}\PY{n}{i}\PY{o}{\PYZpc{}}\PY{k}{2}):
                  \PY{n}{decimg\PYZus{}a} \PY{o}{=} \PY{n}{img\PYZus{}as\PYZus{}float}\PY{p}{(}\PY{n}{color}\PY{o}{.}\PY{n}{rgb2grey}\PY{p}{(}\PY{n}{decimate\PYZus{}image}\PY{p}{(}\PY{n}{imgset}\PY{p}{[}\PY{n}{i}\PY{p}{]}\PY{p}{,} \PY{l+m+mi}{5}\PY{p}{)}\PY{p}{)}\PY{p}{)} \PY{c+c1}{\PYZsh{} downsample to make it easier to see graphs}
                  \PY{n}{decimg\PYZus{}b} \PY{o}{=} \PY{n}{img\PYZus{}as\PYZus{}float}\PY{p}{(}\PY{n}{color}\PY{o}{.}\PY{n}{rgb2grey}\PY{p}{(}\PY{n}{decimate\PYZus{}image}\PY{p}{(}\PY{n}{imgset}\PY{p}{[}\PY{p}{(}\PY{n}{i}\PY{o}{+}\PY{l+m+mi}{1}\PY{p}{)} \PY{o}{\PYZpc{}} \PY{n+nb}{len}\PY{p}{(}\PY{n}{imgset}\PY{p}{)}\PY{p}{]}\PY{p}{,} \PY{l+m+mi}{5}\PY{p}{)}\PY{p}{)}\PY{p}{)}
                  \PY{n}{a}\PY{p}{,} \PY{n}{b}\PY{p}{,} \PY{n}{d} \PY{o}{=} \PY{n}{find\PYZus{}pairwise\PYZus{}difference}\PY{p}{(}\PY{n}{decimg\PYZus{}a}\PY{p}{,} \PY{n}{decimg\PYZus{}b}\PY{p}{)}
                  \PY{n}{plt}\PY{o}{.}\PY{n}{figure}\PY{p}{(}\PY{p}{)}
                  \PY{n}{plt}\PY{o}{.}\PY{n}{subplot}\PY{p}{(}\PY{l+m+mi}{1}\PY{p}{,} \PY{l+m+mi}{3}\PY{p}{,} \PY{l+m+mi}{1}\PY{p}{)}
                  \PY{n}{plt}\PY{o}{.}\PY{n}{title}\PY{p}{(}\PY{l+s+s1}{\PYZsq{}}\PY{l+s+s1}{Claredon}\PY{l+s+s1}{\PYZsq{}}\PY{p}{)}
                  \PY{n}{plt}\PY{o}{.}\PY{n}{imshow}\PY{p}{(}\PY{n}{a}\PY{p}{)}
                  \PY{n}{plt}\PY{o}{.}\PY{n}{subplot}\PY{p}{(}\PY{l+m+mi}{1}\PY{p}{,} \PY{l+m+mi}{3}\PY{p}{,} \PY{l+m+mi}{2}\PY{p}{)}
                  \PY{n}{plt}\PY{o}{.}\PY{n}{title}\PY{p}{(}\PY{l+s+s1}{\PYZsq{}}\PY{l+s+s1}{Original}\PY{l+s+s1}{\PYZsq{}}\PY{p}{)}
                  \PY{n}{plt}\PY{o}{.}\PY{n}{imshow}\PY{p}{(}\PY{n}{b}\PY{p}{)}
                  \PY{n}{plt}\PY{o}{.}\PY{n}{subplot}\PY{p}{(}\PY{l+m+mi}{1}\PY{p}{,} \PY{l+m+mi}{3}\PY{p}{,} \PY{l+m+mi}{3}\PY{p}{)}
                  \PY{n}{plt}\PY{o}{.}\PY{n}{title}\PY{p}{(}\PY{l+s+s1}{\PYZsq{}}\PY{l+s+s1}{Difference}\PY{l+s+s1}{\PYZsq{}}\PY{p}{)}
                  \PY{n}{plt}\PY{o}{.}\PY{n}{imshow}\PY{p}{(}\PY{n}{d}\PY{p}{)}
\end{Verbatim}

    \begin{center}
    \adjustimage{max size={0.9\linewidth}{0.9\paperheight}}{Instagram_Reverse_Engineer_files/Instagram_Reverse_Engineer_14_0.png}
    \end{center}
    { \hspace*{\fill} \\}
    
    \begin{center}
    \adjustimage{max size={0.9\linewidth}{0.9\paperheight}}{Instagram_Reverse_Engineer_files/Instagram_Reverse_Engineer_14_1.png}
    \end{center}
    { \hspace*{\fill} \\}
    
    \begin{center}
    \adjustimage{max size={0.9\linewidth}{0.9\paperheight}}{Instagram_Reverse_Engineer_files/Instagram_Reverse_Engineer_14_2.png}
    \end{center}
    { \hspace*{\fill} \\}
    
    Here we look at the pairwise differences between the images and their
filtered counterparts by looking at them pixel by pixel. There appears
to be sharpening happening because the edges are clearly more different
than the other parts of the image.

    \subsubsection{iii. Add Contrast}\label{iii.-add-contrast}

    We saw from the histograms that is contrasting in play, we try to add
some contrast to the image

    \begin{Verbatim}[commandchars=\\\{\}]
{\color{incolor}In [{\color{incolor}356}]:} \PY{n}{origs} \PY{o}{=} \PY{p}{[}\PY{p}{]}
          \PY{n}{origs}\PY{o}{.}\PY{n}{append}\PY{p}{(}\PY{n}{Image}\PY{o}{.}\PY{n}{open}\PY{p}{(}\PY{l+s+s2}{\PYZdq{}}\PY{l+s+s2}{./images/acai\PYZus{}orig.JPG}\PY{l+s+s2}{\PYZdq{}}\PY{p}{)}\PY{p}{)}
          \PY{n}{origs}\PY{o}{.}\PY{n}{append}\PY{p}{(}\PY{n}{Image}\PY{o}{.}\PY{n}{open}\PY{p}{(}\PY{l+s+s2}{\PYZdq{}}\PY{l+s+s2}{./images/apt\PYZus{}orig.JPG}\PY{l+s+s2}{\PYZdq{}}\PY{p}{)}\PY{p}{)}
          \PY{n}{origs}\PY{o}{.}\PY{n}{append}\PY{p}{(}\PY{n}{Image}\PY{o}{.}\PY{n}{open}\PY{p}{(}\PY{l+s+s2}{\PYZdq{}}\PY{l+s+s2}{./images/aus\PYZus{}orig.JPG}\PY{l+s+s2}{\PYZdq{}}\PY{p}{)}\PY{p}{)}
\end{Verbatim}

    \begin{Verbatim}[commandchars=\\\{\}]
{\color{incolor}In [{\color{incolor}357}]:} \PY{c+c1}{\PYZsh{} Apply a contrast enhancement filter to a copy of this image}
          \PY{k}{def} \PY{n+nf}{contrast}\PY{p}{(}\PY{n}{img}\PY{p}{)}\PY{p}{:}
              \PY{n}{copy} \PY{o}{=} \PY{n}{img}\PY{o}{.}\PY{n}{copy}\PY{p}{(}\PY{p}{)}
              \PY{n}{enhancer} \PY{o}{=} \PY{n}{ImageEnhance}\PY{o}{.}\PY{n}{Contrast}\PY{p}{(}\PY{n}{copy}\PY{p}{)}
              \PY{k}{return} \PY{n}{copy}
\end{Verbatim}

    \begin{Verbatim}[commandchars=\\\{\}]
{\color{incolor}In [{\color{incolor}305}]:} \PY{c+c1}{\PYZsh{} Apply contrasting to all original images}
          \PY{k}{for} \PY{n}{i} \PY{o+ow}{in} \PY{n+nb}{range}\PY{p}{(}\PY{l+m+mi}{3}\PY{p}{)}\PY{p}{:}
              \PY{n}{plt}\PY{o}{.}\PY{n}{figure}\PY{p}{(}\PY{p}{)}
              \PY{n}{plt}\PY{o}{.}\PY{n}{subplot}\PY{p}{(}\PY{l+m+mi}{1}\PY{p}{,} \PY{l+m+mi}{2}\PY{p}{,} \PY{l+m+mi}{1}\PY{p}{)}
              \PY{n}{plt}\PY{o}{.}\PY{n}{title}\PY{p}{(}\PY{l+s+s1}{\PYZsq{}}\PY{l+s+s1}{Original with enhanced contrast}\PY{l+s+s1}{\PYZsq{}}\PY{p}{)}
              \PY{n}{img} \PY{o}{=} \PY{n}{contrast}\PY{p}{(}\PY{n}{origs}\PY{p}{[}\PY{n}{i}\PY{p}{]}\PY{p}{)}
              \PY{n}{plt}\PY{o}{.}\PY{n}{imshow}\PY{p}{(}\PY{n}{img}\PY{p}{)}
\end{Verbatim}

    \begin{center}
    \adjustimage{max size={0.9\linewidth}{0.9\paperheight}}{Instagram_Reverse_Engineer_files/Instagram_Reverse_Engineer_20_0.png}
    \end{center}
    { \hspace*{\fill} \\}
    
    \begin{center}
    \adjustimage{max size={0.9\linewidth}{0.9\paperheight}}{Instagram_Reverse_Engineer_files/Instagram_Reverse_Engineer_20_1.png}
    \end{center}
    { \hspace*{\fill} \\}
    
    \begin{center}
    \adjustimage{max size={0.9\linewidth}{0.9\paperheight}}{Instagram_Reverse_Engineer_files/Instagram_Reverse_Engineer_20_2.png}
    \end{center}
    { \hspace*{\fill} \\}
    
    We see that the contrast adds a boldness that brings us closer to
looking like the Claredon filter.

    \subsubsection{iv. Bring out the blues}\label{iv.-bring-out-the-blues}

    We also noted that there was higher levels on the blue values in the
claredon images. We bring up the blues and reduce the reds.

    \begin{Verbatim}[commandchars=\\\{\}]
{\color{incolor}In [{\color{incolor}306}]:} \PY{k}{def} \PY{n+nf}{blueup}\PY{p}{(}\PY{n}{img}\PY{p}{)}\PY{p}{:}
              \PY{c+c1}{\PYZsh{} See Reference [1]}
              \PY{n}{imgcopy} \PY{o}{=} \PY{n}{img}\PY{o}{.}\PY{n}{copy}\PY{p}{(}\PY{p}{)}
              \PY{c+c1}{\PYZsh{}adjusting the red lower red values to be diminished}
              \PY{n}{imgcopy}\PY{p}{[}\PY{n}{imgcopy}\PY{p}{[}\PY{p}{:}\PY{p}{,}\PY{p}{:}\PY{p}{,}\PY{l+m+mi}{0}\PY{p}{]} \PY{o}{\PYZlt{}}\PY{o}{=} \PY{l+m+mf}{0.05}\PY{p}{]} \PY{o}{/}\PY{o}{=} \PY{l+m+mi}{2}
              \PY{c+c1}{\PYZsh{}adjusting the green lower values to be diminished}
              \PY{n}{imgcopy}\PY{p}{[}\PY{n}{imgcopy}\PY{p}{[}\PY{p}{:}\PY{p}{,}\PY{p}{:}\PY{p}{,}\PY{l+m+mi}{1}\PY{p}{]} \PY{o}{\PYZlt{}}\PY{o}{=} \PY{l+m+mf}{0.05}\PY{p}{]} \PY{o}{/}\PY{o}{=} \PY{l+m+mi}{2}
              \PY{c+c1}{\PYZsh{}adjusting the blue values to be increased}
              \PY{n}{imgcopy}\PY{p}{[}\PY{p}{:}\PY{p}{,}\PY{p}{:}\PY{p}{,}\PY{l+m+mi}{2}\PY{p}{]} \PY{o}{+}\PY{o}{=} \PY{l+m+mf}{0.08}
              \PY{n}{imgcopy}\PY{p}{[}\PY{p}{:}\PY{p}{,}\PY{p}{:}\PY{p}{,}\PY{l+m+mi}{2}\PY{p}{]} \PY{o}{/}\PY{o}{=} \PY{l+m+mi}{2}
              \PY{k}{return} \PY{n}{img}
\end{Verbatim}

    \begin{Verbatim}[commandchars=\\\{\}]
{\color{incolor}In [{\color{incolor}337}]:} \PY{k}{for} \PY{n}{i}\PY{p}{,}\PY{n}{img} \PY{o+ow}{in} \PY{n+nb}{enumerate}\PY{p}{(}\PY{n}{imgset2}\PY{p}{)}\PY{p}{:}
              \PY{k}{if} \PY{p}{(}\PY{l+m+mi}{1} \PY{o}{==} \PY{n}{i}\PY{o}{\PYZpc{}}\PY{k}{2}):
                  \PY{n}{plt}\PY{o}{.}\PY{n}{figure}\PY{p}{(}\PY{p}{)}
                  \PY{n}{plt}\PY{o}{.}\PY{n}{subplot}\PY{p}{(}\PY{l+m+mi}{1}\PY{p}{,} \PY{l+m+mi}{2}\PY{p}{,} \PY{l+m+mi}{1}\PY{p}{)}
                  \PY{n}{plt}\PY{o}{.}\PY{n}{title}\PY{p}{(}\PY{l+s+s1}{\PYZsq{}}\PY{l+s+s1}{Original with color adjustment}\PY{l+s+s1}{\PYZsq{}}\PY{p}{)}
                  \PY{n}{blue} \PY{o}{=} \PY{n}{blueup}\PY{p}{(}\PY{n}{img}\PY{p}{)}
                  \PY{n}{plt}\PY{o}{.}\PY{n}{imshow}\PY{p}{(}\PY{n}{blue}\PY{p}{)}
                  \PY{n}{skimage}\PY{o}{.}\PY{n}{io}\PY{o}{.}\PY{n}{imsave}\PY{p}{(}\PY{l+s+s2}{\PYZdq{}}\PY{l+s+s2}{./images/blued}\PY{l+s+si}{\PYZpc{}d}\PY{l+s+s2}{.JPG}\PY{l+s+s2}{\PYZdq{}} \PY{o}{\PYZpc{}} \PY{n}{i}\PY{p}{,} \PY{n}{blue}\PY{p}{)}
\end{Verbatim}

    \begin{Verbatim}[commandchars=\\\{\}]
/Users/Lisa/anaconda/lib/python3.5/site-packages/skimage/util/dtype.py:110: UserWarning: Possible precision loss when converting from float64 to uint8
  "\%s to \%s" \% (dtypeobj\_in, dtypeobj))

    \end{Verbatim}

    \begin{center}
    \adjustimage{max size={0.9\linewidth}{0.9\paperheight}}{Instagram_Reverse_Engineer_files/Instagram_Reverse_Engineer_25_1.png}
    \end{center}
    { \hspace*{\fill} \\}
    
    \begin{center}
    \adjustimage{max size={0.9\linewidth}{0.9\paperheight}}{Instagram_Reverse_Engineer_files/Instagram_Reverse_Engineer_25_2.png}
    \end{center}
    { \hspace*{\fill} \\}
    
    \begin{center}
    \adjustimage{max size={0.9\linewidth}{0.9\paperheight}}{Instagram_Reverse_Engineer_files/Instagram_Reverse_Engineer_25_3.png}
    \end{center}
    { \hspace*{\fill} \\}
    
    These values appear to be more yellow than the Claredon filter. In an
unrelated note: these images have the appearance of an image viewed on a
smartphone that is in ``Night Mode''

    \subsubsection{v. Sharpen}\label{v.-sharpen}

    We noted that sharpening likely took place. Here we apply some
sharpening to the images to get closer to the filtered images.

    \begin{Verbatim}[commandchars=\\\{\}]
{\color{incolor}In [{\color{incolor}325}]:} \PY{c+c1}{\PYZsh{}sharpen the image}
          \PY{k}{def} \PY{n+nf}{sharpener}\PY{p}{(}\PY{n}{img}\PY{p}{)}\PY{p}{:}
              \PY{n}{img}\PY{o}{.}\PY{n}{convert}\PY{p}{(}\PY{l+s+s1}{\PYZsq{}}\PY{l+s+s1}{RGB}\PY{l+s+s1}{\PYZsq{}}\PY{p}{)}\PY{o}{.}\PY{n}{filter}\PY{p}{(}\PY{n}{ImageFilter}\PY{o}{.}\PY{n}{SHARPEN}\PY{p}{)}
              \PY{k}{return} \PY{n}{img}
\end{Verbatim}

    \begin{Verbatim}[commandchars=\\\{\}]
{\color{incolor}In [{\color{incolor}309}]:} \PY{c+c1}{\PYZsh{}apply sharpening to each image}
          \PY{k}{for} \PY{n}{i} \PY{o+ow}{in} \PY{n+nb}{range}\PY{p}{(}\PY{l+m+mi}{3}\PY{p}{)}\PY{p}{:}
                  \PY{n}{plt}\PY{o}{.}\PY{n}{figure}\PY{p}{(}\PY{p}{)}
                  \PY{n}{plt}\PY{o}{.}\PY{n}{subplot}\PY{p}{(}\PY{l+m+mi}{1}\PY{p}{,} \PY{l+m+mi}{2}\PY{p}{,} \PY{l+m+mi}{2}\PY{p}{)}
                  \PY{n}{plt}\PY{o}{.}\PY{n}{title}\PY{p}{(}\PY{l+s+s1}{\PYZsq{}}\PY{l+s+s1}{Original with sharpening}\PY{l+s+s1}{\PYZsq{}}\PY{p}{)}
                  \PY{n}{plt}\PY{o}{.}\PY{n}{imshow}\PY{p}{(}\PY{n}{sharpener}\PY{p}{(}\PY{n}{origs}\PY{p}{[}\PY{n}{i}\PY{p}{]}\PY{p}{)}\PY{p}{)}
\end{Verbatim}

    \begin{center}
    \adjustimage{max size={0.9\linewidth}{0.9\paperheight}}{Instagram_Reverse_Engineer_files/Instagram_Reverse_Engineer_30_0.png}
    \end{center}
    { \hspace*{\fill} \\}
    
    \begin{center}
    \adjustimage{max size={0.9\linewidth}{0.9\paperheight}}{Instagram_Reverse_Engineer_files/Instagram_Reverse_Engineer_30_1.png}
    \end{center}
    { \hspace*{\fill} \\}
    
    \begin{center}
    \adjustimage{max size={0.9\linewidth}{0.9\paperheight}}{Instagram_Reverse_Engineer_files/Instagram_Reverse_Engineer_30_2.png}
    \end{center}
    { \hspace*{\fill} \\}
    
    The sharpening seems to bring the images into focus and improves
clarity.

    \subsection{Part D - Results}\label{part-d---results}

    \subsubsection{i. Visual comparison between our filter and
Instagram's}\label{i.-visual-comparison-between-our-filter-and-instagrams}

    After performing the enhancements separately, we bring them together.

    \begin{Verbatim}[commandchars=\\\{\}]
{\color{incolor}In [{\color{incolor}347}]:} \PY{n}{j} \PY{o}{=} \PY{l+m+mi}{0}
          \PY{n}{filtered} \PY{o}{=} \PY{p}{[}\PY{p}{]}
          \PY{k}{for} \PY{n}{i} \PY{o+ow}{in} \PY{n+nb}{range}\PY{p}{(}\PY{l+m+mi}{3}\PY{p}{)}\PY{p}{:}
              \PY{n}{plt}\PY{o}{.}\PY{n}{figure}\PY{p}{(}\PY{p}{)}
              \PY{n}{plt}\PY{o}{.}\PY{n}{subplot}\PY{p}{(}\PY{l+m+mi}{1}\PY{p}{,} \PY{l+m+mi}{3}\PY{p}{,} \PY{l+m+mi}{1}\PY{p}{)}
              \PY{n}{plt}\PY{o}{.}\PY{n}{title}\PY{p}{(}\PY{l+s+s1}{\PYZsq{}}\PY{l+s+s1}{Original}\PY{l+s+s1}{\PYZsq{}}\PY{p}{)}
              \PY{n}{plt}\PY{o}{.}\PY{n}{imshow}\PY{p}{(}\PY{n}{origs}\PY{p}{[}\PY{n}{i}\PY{p}{]}\PY{p}{)}
              \PY{n}{plt}\PY{o}{.}\PY{n}{subplot}\PY{p}{(}\PY{l+m+mi}{1}\PY{p}{,} \PY{l+m+mi}{3}\PY{p}{,} \PY{l+m+mi}{2}\PY{p}{)}
              \PY{n}{plt}\PY{o}{.}\PY{n}{title}\PY{p}{(}\PY{l+s+s1}{\PYZsq{}}\PY{l+s+s1}{Claredon}\PY{l+s+s1}{\PYZsq{}}\PY{p}{)}
              \PY{n}{plt}\PY{o}{.}\PY{n}{imshow}\PY{p}{(}\PY{n}{imgset}\PY{p}{[}\PY{n}{j}\PY{p}{]}\PY{p}{)}
              \PY{n}{plt}\PY{o}{.}\PY{n}{subplot}\PY{p}{(}\PY{l+m+mi}{1}\PY{p}{,} \PY{l+m+mi}{3}\PY{p}{,} \PY{l+m+mi}{3}\PY{p}{)}
              \PY{n}{plt}\PY{o}{.}\PY{n}{title}\PY{p}{(}\PY{l+s+s1}{\PYZsq{}}\PY{l+s+s1}{Our Filter}\PY{l+s+s1}{\PYZsq{}}\PY{p}{)}
              \PY{n}{im\PYZus{}blue} \PY{o}{=} \PY{n}{Image}\PY{o}{.}\PY{n}{open}\PY{p}{(}\PY{l+s+s2}{\PYZdq{}}\PY{l+s+s2}{./images/blued}\PY{l+s+si}{\PYZpc{}s}\PY{l+s+s2}{.JPG}\PY{l+s+s2}{\PYZdq{}} \PY{o}{\PYZpc{}} \PY{n+nb}{str}\PY{p}{(}\PY{n}{j}\PY{o}{+}\PY{l+m+mi}{1}\PY{p}{)}\PY{p}{)}
              \PY{n}{im\PYZus{}contrast} \PY{o}{=} \PY{n}{contrast}\PY{p}{(}\PY{n}{im\PYZus{}blue}\PY{p}{)}
              \PY{n}{im\PYZus{}sharpened} \PY{o}{=} \PY{n}{sharpener}\PY{p}{(}\PY{n}{im\PYZus{}contrast}\PY{p}{)}
              \PY{n}{plt}\PY{o}{.}\PY{n}{imshow}\PY{p}{(}\PY{n}{im\PYZus{}sharpened}\PY{p}{)}
              \PY{n}{filtered}\PY{o}{.}\PY{n}{append}\PY{p}{(}\PY{n}{im\PYZus{}sharpened}\PY{p}{)}
              \PY{n}{j} \PY{o}{+}\PY{o}{=} \PY{l+m+mi}{2}
\end{Verbatim}

    \begin{center}
    \adjustimage{max size={0.9\linewidth}{0.9\paperheight}}{Instagram_Reverse_Engineer_files/Instagram_Reverse_Engineer_35_0.png}
    \end{center}
    { \hspace*{\fill} \\}
    
    \begin{center}
    \adjustimage{max size={0.9\linewidth}{0.9\paperheight}}{Instagram_Reverse_Engineer_files/Instagram_Reverse_Engineer_35_1.png}
    \end{center}
    { \hspace*{\fill} \\}
    
    \begin{center}
    \adjustimage{max size={0.9\linewidth}{0.9\paperheight}}{Instagram_Reverse_Engineer_files/Instagram_Reverse_Engineer_35_2.png}
    \end{center}
    { \hspace*{\fill} \\}
    
    \subsubsection{ii. Histogram comparison between our filter and
Instagram's}\label{ii.-histogram-comparison-between-our-filter-and-instagrams}

    \begin{Verbatim}[commandchars=\\\{\}]
{\color{incolor}In [{\color{incolor}350}]:} \PY{c+c1}{\PYZsh{} Apply histograms to our filter and the claredon filter side by side.}
          \PY{n}{k} \PY{o}{=} \PY{l+m+mi}{0}
          \PY{k}{for} \PY{n}{i}\PY{p}{,}\PY{n}{img} \PY{o+ow}{in} \PY{n+nb}{enumerate}\PY{p}{(}\PY{n}{imgset}\PY{p}{)}\PY{p}{:}
                  \PY{k}{if} \PY{p}{(}\PY{l+m+mi}{0} \PY{o}{==} \PY{n}{i}\PY{o}{\PYZpc{}}\PY{k}{2}):
                      \PY{n}{imtitle} \PY{o}{=} \PY{l+s+s1}{\PYZsq{}}\PY{l+s+s1}{Claredon}\PY{l+s+s1}{\PYZsq{}}
                      \PY{n}{plt}\PY{o}{.}\PY{n}{figure}\PY{p}{(}\PY{p}{)}
                      \PY{n}{plt}\PY{o}{.}\PY{n}{subplot}\PY{p}{(}\PY{l+m+mi}{1}\PY{p}{,} \PY{l+m+mi}{4}\PY{p}{,} \PY{l+m+mi}{1}\PY{p}{)}
                      \PY{n}{plt}\PY{o}{.}\PY{n}{title}\PY{p}{(}\PY{n}{imtitle}\PY{p}{)}
                      \PY{n}{plt}\PY{o}{.}\PY{n}{imshow}\PY{p}{(}\PY{n}{img}\PY{p}{,} \PY{n}{cmap}\PY{o}{=}\PY{l+s+s1}{\PYZsq{}}\PY{l+s+s1}{gray}\PY{l+s+s1}{\PYZsq{}}\PY{p}{)}
                      \PY{n}{plt}\PY{o}{.}\PY{n}{subplot}\PY{p}{(}\PY{l+m+mi}{1}\PY{p}{,} \PY{l+m+mi}{4}\PY{p}{,} \PY{l+m+mi}{2}\PY{p}{)}
                      \PY{n}{plt}\PY{o}{.}\PY{n}{title}\PY{p}{(}\PY{n}{imtitle} \PY{o}{+} \PY{l+s+s1}{\PYZsq{}}\PY{l+s+s1}{ Histo}\PY{l+s+s1}{\PYZsq{}}\PY{p}{)}
                      \PY{n}{plot\PYZus{}multichannel\PYZus{}histo}\PY{p}{(}\PY{n}{img}\PY{p}{)}
                      \PY{n}{plt}\PY{o}{.}\PY{n}{subplot}\PY{p}{(}\PY{l+m+mi}{1}\PY{p}{,} \PY{l+m+mi}{4}\PY{p}{,} \PY{l+m+mi}{3}\PY{p}{)}
                      \PY{n}{plt}\PY{o}{.}\PY{n}{title}\PY{p}{(}\PY{l+s+s1}{\PYZsq{}}\PY{l+s+s1}{Our Filter}\PY{l+s+s1}{\PYZsq{}}\PY{p}{)}
                      \PY{n}{plt}\PY{o}{.}\PY{n}{imshow}\PY{p}{(}\PY{n}{filtered}\PY{p}{[}\PY{n}{k}\PY{p}{]}\PY{p}{)}
                      \PY{n}{plt}\PY{o}{.}\PY{n}{subplot}\PY{p}{(}\PY{l+m+mi}{1}\PY{p}{,} \PY{l+m+mi}{4}\PY{p}{,} \PY{l+m+mi}{4}\PY{p}{)}
                      \PY{n}{plt}\PY{o}{.}\PY{n}{title}\PY{p}{(}\PY{l+s+s1}{\PYZsq{}}\PY{l+s+s1}{Our Filter Histo}\PY{l+s+s1}{\PYZsq{}}\PY{p}{)}
                      \PY{n}{plot\PYZus{}multichannel\PYZus{}histo}\PY{p}{(}\PY{n}{img\PYZus{}as\PYZus{}ubyte}\PY{p}{(}\PY{n}{filtered}\PY{p}{[}\PY{n}{k}\PY{p}{]}\PY{p}{)}\PY{p}{)}
                      \PY{n}{k} \PY{o}{=} \PY{n}{k} \PY{o}{+} \PY{l+m+mi}{1}
\end{Verbatim}

    \begin{center}
    \adjustimage{max size={0.9\linewidth}{0.9\paperheight}}{Instagram_Reverse_Engineer_files/Instagram_Reverse_Engineer_37_0.png}
    \end{center}
    { \hspace*{\fill} \\}
    
    \begin{center}
    \adjustimage{max size={0.9\linewidth}{0.9\paperheight}}{Instagram_Reverse_Engineer_files/Instagram_Reverse_Engineer_37_1.png}
    \end{center}
    { \hspace*{\fill} \\}
    
    \begin{center}
    \adjustimage{max size={0.9\linewidth}{0.9\paperheight}}{Instagram_Reverse_Engineer_files/Instagram_Reverse_Engineer_37_2.png}
    \end{center}
    { \hspace*{\fill} \\}
    
    \subsubsection{iii. Pairwise differences between our filter and
Instagram's}\label{iii.-pairwise-differences-between-our-filter-and-instagrams}

    \begin{Verbatim}[commandchars=\\\{\}]
{\color{incolor}In [{\color{incolor}353}]:} \PY{c+c1}{\PYZsh{} Show a pairwise difference between our filter and Instagram\PYZsq{}s filter.}
          \PY{n}{l} \PY{o}{=} \PY{l+m+mi}{0}
          \PY{k}{for} \PY{n}{i} \PY{o+ow}{in} \PY{n+nb}{range}\PY{p}{(}\PY{l+m+mi}{3}\PY{p}{)}\PY{p}{:}
              \PY{n}{decimg\PYZus{}a} \PY{o}{=} \PY{n}{img\PYZus{}as\PYZus{}float}\PY{p}{(}\PY{n}{color}\PY{o}{.}\PY{n}{rgb2grey}\PY{p}{(}\PY{n}{decimate\PYZus{}image}\PY{p}{(}\PY{n}{img\PYZus{}as\PYZus{}ubyte}\PY{p}{(}\PY{n}{filtered}\PY{p}{[}\PY{n}{i}\PY{p}{]}\PY{p}{)}\PY{p}{,} \PY{l+m+mi}{5}\PY{p}{)}\PY{p}{)}\PY{p}{)} \PY{c+c1}{\PYZsh{} downsample to make it easier to see graphs}
              \PY{n}{decimg\PYZus{}b} \PY{o}{=} \PY{n}{img\PYZus{}as\PYZus{}float}\PY{p}{(}\PY{n}{color}\PY{o}{.}\PY{n}{rgb2grey}\PY{p}{(}\PY{n}{decimate\PYZus{}image}\PY{p}{(}\PY{n}{imgset}\PY{p}{[}\PY{p}{(}\PY{n}{l}\PY{p}{)}\PY{p}{]}\PY{p}{,} \PY{l+m+mi}{5}\PY{p}{)}\PY{p}{)}\PY{p}{)}
              \PY{n}{a}\PY{p}{,} \PY{n}{b}\PY{p}{,} \PY{n}{d} \PY{o}{=} \PY{n}{find\PYZus{}pairwise\PYZus{}difference}\PY{p}{(}\PY{n}{decimg\PYZus{}a}\PY{p}{,} \PY{n}{decimg\PYZus{}b}\PY{p}{)}
              \PY{n}{l} \PY{o}{=} \PY{n}{l} \PY{o}{+} \PY{l+m+mi}{2}
              \PY{n}{plt}\PY{o}{.}\PY{n}{figure}\PY{p}{(}\PY{p}{)}
              \PY{n}{plt}\PY{o}{.}\PY{n}{subplot}\PY{p}{(}\PY{l+m+mi}{1}\PY{p}{,} \PY{l+m+mi}{3}\PY{p}{,} \PY{l+m+mi}{1}\PY{p}{)}
              \PY{n}{plt}\PY{o}{.}\PY{n}{title}\PY{p}{(}\PY{l+s+s1}{\PYZsq{}}\PY{l+s+s1}{Our Filter}\PY{l+s+s1}{\PYZsq{}}\PY{p}{)}
              \PY{n}{plt}\PY{o}{.}\PY{n}{imshow}\PY{p}{(}\PY{n}{a}\PY{p}{)}
              \PY{n}{plt}\PY{o}{.}\PY{n}{subplot}\PY{p}{(}\PY{l+m+mi}{1}\PY{p}{,} \PY{l+m+mi}{3}\PY{p}{,} \PY{l+m+mi}{2}\PY{p}{)}
              \PY{n}{plt}\PY{o}{.}\PY{n}{title}\PY{p}{(}\PY{l+s+s1}{\PYZsq{}}\PY{l+s+s1}{Claredon}\PY{l+s+s1}{\PYZsq{}}\PY{p}{)}
              \PY{n}{plt}\PY{o}{.}\PY{n}{imshow}\PY{p}{(}\PY{n}{b}\PY{p}{)}
              \PY{n}{plt}\PY{o}{.}\PY{n}{subplot}\PY{p}{(}\PY{l+m+mi}{1}\PY{p}{,} \PY{l+m+mi}{3}\PY{p}{,} \PY{l+m+mi}{3}\PY{p}{)}
              \PY{n}{plt}\PY{o}{.}\PY{n}{title}\PY{p}{(}\PY{l+s+s1}{\PYZsq{}}\PY{l+s+s1}{Difference}\PY{l+s+s1}{\PYZsq{}}\PY{p}{)}
              \PY{n}{plt}\PY{o}{.}\PY{n}{imshow}\PY{p}{(}\PY{n}{d}\PY{p}{)}
\end{Verbatim}

    \begin{center}
    \adjustimage{max size={0.9\linewidth}{0.9\paperheight}}{Instagram_Reverse_Engineer_files/Instagram_Reverse_Engineer_39_0.png}
    \end{center}
    { \hspace*{\fill} \\}
    
    \begin{center}
    \adjustimage{max size={0.9\linewidth}{0.9\paperheight}}{Instagram_Reverse_Engineer_files/Instagram_Reverse_Engineer_39_1.png}
    \end{center}
    { \hspace*{\fill} \\}
    
    \begin{center}
    \adjustimage{max size={0.9\linewidth}{0.9\paperheight}}{Instagram_Reverse_Engineer_files/Instagram_Reverse_Engineer_39_2.png}
    \end{center}
    { \hspace*{\fill} \\}
    
    \subsection{Part E - Discussion}\label{part-e---discussion}

    Clearly, our filter is vastly different from the Claredon filter. Ours
contains much more yellow and green tones than the claredon filter,
which seems to somehow pull out aqua and indigo colors much more
effectively than us. As we can see from the histograms, lower blue
values on ours are much higher, while higher blue values on theirs are
higher. Additionally, as we can see in the pairwise difference between
our filter and theirs, it looks like we may have gone overboard on the
sharpening to the point where the edges are almost more different than
when compared to the original. Still, general idea of color
transformation seems to apply, and the use of sharpening and contrast at
least appear to get the boldness of the Claredon filter, even if the
colors are not correct.

    \subsection{Part F - References}\label{part-f---references}

    {[}1{]} Code modified from provided
Nashville\_Instagram\_Reverse\_Engineering

    {[}2{]} Code modified from Assignment 2 - Image Analysis.

    {[}3{]} All images are mine.


    % Add a bibliography block to the postdoc
    
    
    
    \end{document}
